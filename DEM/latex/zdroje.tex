\begin{thebibliography}{9}

\bibitem{Bayer 2008}
BAYER, T. (2008): Algoritmy v digitální kartografii. Nakladatelství Karolinum, Praha.

\bibitem{Bayer 2023}
BAYER, T. (2023a): Rovinné triangulace a jejich využití. Přednáška pro předmět Algoritmy počítačové kartografie, Katedra aplikované geoinformatiky a kartografie. Přírodovědecká fakulta UK, dostupné \href{http://web.natur.cuni.cz/~bayertom/images/courses/Adk/adk5_new.pdf}{\emph{zde}} (cit. 17. 4. 2023).

\bibitem{Bayer 2023b}
BAYER, T. (2023): Úvod do výpočetní geometrie. Základní vztahy. Přednáška pro předmět Algoritmy počítačové kartografie, Katedra aplikované geoinformatiky a kartografie. Přírodovědecká fakulta UK, dostupné \href{http://web.natur.cuni.cz/~bayertom/images/courses/Adk/adk2.pdf}{\emph{zde}} (cit. 17. 4. 2023).

\bibitem{ESRI}
BUCKLEY, A. (2008): Esri Blog, Mapping, Aspect-slope map, dostupné 
\href{https://www.esri.com/arcgis-blog/products/product/mapping/aspect-slope-map/}{\emph{zde}}

\bibitem{de Berg}
DE BERG, M., CHEONG, O., VAN KREVELD, M., OVERMARS, B. (2008): Computational Geometry. Algorithms and Applications, third edition. Springer, Berlin.

\bibitem{Bruha}
BRŮHA, L. (2016): DIGITÁLNÍ	MODELY	TERÉNU, výukový materiál, verze 1.0. PřF UK, Praha.

\bibitem{Rouke}
ROURKE, O. J. (2005): Computational Geometry in C. Cambridge University Press, Cambridge.

\bibitem{Zara}
ŽÁRA, J., BEDŘICH, B., SOCHOR, J., FELKEL, P. (2004): Moderní počítačová grafika. Computer Press, Brno.
\end{thebibliography}