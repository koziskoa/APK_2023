\chapter*{Zadání}

\par \emph {Vstup: množina P} $= \{p_1, \dots, p_n\}$, $p_i = \{x_i, y_i, z_i\}$.
\par \emph {Výstup: polyedrický \textbf{DMT} nad množinou $P$ představovaný vrstevnicemi doplněný vizualizací sklonu trojúhelníků a jejich expozicí.}.

\par Metodou inkrementální konstrukce vytvořte nad množinou P vstupních bodů 2D \emph{Delaunay triangulaci}. Jako vstupní data použijte existující geodetická data (alespoň 300 bodů) popř. navrhněte algoritmus pro generování syntetických vstupních dat představujících významné terénní tvary (kupa, údolí, spočinek, hřbet, \dots).

\par Vstupní množiny bodů včetně níže uvedených výstupů vhodně vizualizujte. Grafické rozhraní realizujte s využitím frameworku QT. Dynamické datové struktury implementujte s využitím STL.

\par Nad takto vzniklou triangulací vygenerujte polyedrický digitální model terénu. Dále proveďte tyto analýzy:

\begin{itemize}
  \item S využitím lineární interpolace vygenerujte vrstevnice se zadaným krokem a v zadaném intervalu, proveďte jejich vizualizaci s rozlišením zvýrazněných vrstevnic.
  \item Analyzujte sklon digitálního modelu terénu, jednotlivé trojúhelníky vizualizujte v závislosti na jejich sklonu.
  \item Analyzujte expozici digitálního modelu terénu, jednotlivé trojúhelníky vizualizujte v závislosti na jejich expozici ke světové straně.
\end{itemize}

\par Zhodnoťte výsledný digitální model terénu z kartografického hlediska, zamyslete se nad slabinami algoritmu založeného na 2D Delaunay triangulaci. Ve kterých situacích (různé terénní tvary) nebude dávat vhodné výsledky? Tyto situace graficky znázorněte.

\par Zhodnocení činnosti algoritmu včetně ukázek proveďte alespoň na 3 strany formátu A4.
\bigbreak

% insert table
\par\textbf{\large Hodnocení:}
%\bigbreak
\begin{center}
    \begin{tabular}{|p{14.2cm}|c|} 
     \hline\large
         \textbf{Krok} & \textbf{Hodnocení}\\ %[0.5ex] 
             \hline\hline
             \small \emph{Delaunay triangulace, polyedrický model terénu.}  & \emph{\small10b} \\ 
             \hline
             \emph{\small Konstrukce vrstevnic, analýza sklonu a expozice.} & \emph{\small 10b} \\
             \hline
             \emph{\small Triangulace nekonvexní oblasti zadané polygonem} & \emph{\small +5b} \\
             \hline
             \emph{\small Výběr barevných stupnic při vizualizaci sklonu a expozice.} & \emph{\small +3b} \\
             \hline
             \emph{\small Automatický popis vrstevnic.} & \emph{\small +3b}\\ 
             \hline
             \emph{\small Automatický popis vrstevnic respektující kartografické zásady (orientace, vhodné rozložení).} & \emph{\small +10b}\\ 
             \hline
             \emph{\small Algoritmus pro automatické generování terénních tvarů (kupa, údolí, spočinek, hřbet, \dots).} & \emph{\small +10b}\\ 
             \hline
             \emph{\small 3D vizualizace terénu s využitím promítání.} & \emph{\small +10b}\\ 
             \hline
             \emph{\small Barevná hypsometrie} & \emph{\small +5b}\\ 
             \hline
             \textbf{Max celkem:} & \textbf{65b}\\ %[0.3ex] 
             \hline
    \end{tabular}
\end{center}
