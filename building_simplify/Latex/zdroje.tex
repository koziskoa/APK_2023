\begin{thebibliography}{9}

\bibitem{Bayer 2023}
BAYER, T. (2023a): Konvexní obálka množiny bodů. Přednáška pro předmět Algoritmy počítačové kartografie, Katedra aplikované geoinformatiky a kartografie. Přírodovědecká fakulta UK, dostupné \href{http://web.natur.cuni.cz/~bayertom/images/courses/Adk/adk4_new.pdf}{\emph{zde}} (cit. 30. 3. 2023).

\bibitem{Bayer 2023b}
BAYER, T. (2023): Úvod do výpočetní geometrie. Základní vztahy. Přednáška pro předmět Algoritmy počítačové kartografie, Katedra aplikované geoinformatiky a kartografie. Přírodovědecká fakulta UK, dostupné \href{http://web.natur.cuni.cz/~bayertom/images/courses/Adk/adk2.pdf}{\emph{zde}} (cit. 30. 3. 2023).

\bibitem{de Berg}
DE BERG, M., CHEONG, O., VAN KREVELD, M., OVERMARS, B. (2008): Computational Geometry. Algorithms and Applications, third edition. Springer, Berlin.

\bibitem{Eberly}
EBERLEY, D. (2020): Minimum-Area Rectangle Containing a Set of Points. Geometric Tools, Redmond WA 98052, dostupné \href{https://www.geometrictools.com/Documentation/MinimumAreaRectangle.pdf}{\emph{zde}} (cit. 30. 3. 2023).


\bibitem{Monmonier}
MONMONIER, M. (2018): How to lie with maps, second edition. The University of Chicago Press, Chicago.

\bibitem{Rouke}
ROURKE, O. J. (2005): Computational Geometry in C. Cambridge University Press, Cambridge.

\bibitem{Wood}
WOOD, J. (2008): Minimum Bounding Rectangle. In: Shekhar, S., Xiong, H. (eds.) Encyclopedia of GIS. Springer, Boston.
\end{thebibliography}