\chapter*{Zadání}

\par\emph{Vstup: Souvislá polygonová mapa n polygonů} \{$P_1, ..., P_n$\}, \emph{analyzovaný bod q}.
\par \emph{Výstup: $P_i, q \in P_i$}.

\par Nad polygonovou mapou implementujte Ray Crossing Algorithm pro geometrické vyhledání incidujícího polygonu obsahujícího zadaný bod \emph{q}.
\par Nalezený polygon graficky zvýrazněte vhodným způsobem (např. vyplněním, šrafováním, blikáním). Grafické rozhraní vytvořte s využitím frameworku QT.
\par Pro generování nekonvexních polygonů můžete navrhnout vlastní algoritmus či použít existující geografická data (např. mapa evropských států).
\par Polygony budou načítány z textového souboru ve Vámi zvoleném formátu. Pro datovou reprezentaci jednotlivých polygonů použijte špagetový model.
\bigbreak

% insert table
\par\textbf{\large Hodnocení:}
\bigbreak
\begin{center}
    \begin{tabular}{|p{14.2cm}|c|} 
     \hline\large
         \textbf{Krok} & \textbf{Hodnocení}\\ %[0.5ex] 
             \hline\hline
             \small Detekce polohy bodu rozlišující stavy uvnitř, vně polygonu. & \small10b \\ 
             \hline
             \emph{\small Analýza polohy bodu (uvnitř/vně) metodou Winding Number Algorithm.} & \emph{\small +10b} \\
             \hline
             \emph{\small Ošetření singulárního případu u Winding Number Algorithm: bod leží na hraně polygonu.} & \emph{\small +5b} \\
             \hline
             \emph{\small Ošetření singulárního případu u Ray Crossing Algorithm: bod leží na hraně polygonu.} & \emph{\small +5b} \\
             \hline
             \emph{\small Ošetření singulárního případu u obou algoritmů: bod je totožný s vrcholem jednoho či více polygonů.} & \emph{\small +2b}\\ 
             \hline
             \emph{\small Zvýraznění všech polygonů pro oba výše uvedené singulární případy.} & \emph{\small +3b}\\  
             \hline
             \textbf{Max celkem:} & \textbf{35b}\\ %[0.3ex] 
             \hline
    \end{tabular}
\end{center}