\chapter*{Zadání}

\par S využitím generalizačního operátoru Partial Modification realizujte odsun a částečnou změnu tvaru jednoho blízkého prvku vůči blízkému pevnému prvku (bariéře) tak, aby v generalizované mapě nedošlo k jejich grafickému konfliktu. Hodnotu minimální vzdálenosti prvků \underbar{$d$} volte v závislosti na měřítkovém čísle mapy (např. 1 mm v mapě). Pro implementaci použijte metodu energetických splinů.
\par Jako vstupní data použijte existující kartografická data (např. silniční či železniční síť, vodstvo), která budou načítána ze dvou textových souborů ve vámi zvoleném formátu.
\par Grafické rozhraní realizujte s využitím frameworku QT, výsledky generalizačních operací vizualizujte. Porovnejte dosažené výsledky s ruční generalizací prováděnou kartografickým expertem.
\bigbreak

% insert table
\par\textbf{\large Hodnocení:}
\bigbreak
\begin{center}
    \begin{tabular}{|p{14.2cm}|c|} 
     \hline\large
         \textbf{Krok} & \textbf{Hodnocení}\\ %[0.5ex] 
             \hline\hline
             \small Partial Modification: 1 prvek a překážka. & \small20b \\ 
             \hline
             \emph{\small Partial Modification: 2 blízké prvky.} & \emph{\small +10b} \\
             \hline
             \emph{\small Partial Modification: 2 prvky a překážka.} & \emph{\small +15b} \\
             \hline
             \emph{\small Partial Modification: 3 prvky.} & \emph{\small +15b} \\
             \hline
             \emph{\small Variabilní počet iterací, polohová chyba menší než grafická přesnost mapy.} & \emph{\small +5b}\\ 
             \hline
             \textbf{Max celkem:} & \textbf{55b}\\ %[0.3ex] 
             \hline
    \end{tabular}
\end{center}